\documentclass{article}

\usepackage[letterpaper, left=1in, right=1in]{geometry}
\usepackage{graphicx}
\usepackage{amsmath}
\usepackage[table]{xcolor}
\usepackage{tcolorbox}

\newcommand{\solution}[1]{\begin{tcolorbox}[colback=white,colframe=black!70!white,title=Solution]
#1
\end{tcolorbox}}

\title{\textbf{\huge Habib University}\\[0.5cm]
\Large CE/CS 321/330 Computer Architecture\\[0.5cm]
\large Homework 4\\[0.5cm]
\large Hammad Sajid (hs07606)
\\ \large Muhammad Owais Waheed (ow07611)}
\author{}
\date{}

\begin{document}

\maketitle

\section*{Question 1}
Caches are important to providing a high-performance memory hierarchy to processors. Below is
a list of 64-bit memory address references, given as word addresses.
0x03, 0xb4, 0x2b, 0x02, 0xbf, 0x58, 0xbe, 0x0e, 0xb5, 0x2c, 0xba, 0xfd

\vspace*{0.5 cm}
a. For each of these references, identify the binary word address, the tag, and the
index given a direct-mapped cache with 16 one-word blocks. Also list whether each
reference is a hit or a miss, assuming the cache is initially empty. First insertion is already
done so that you may get the idea.

\begin{table}[h]
    \centering
    \begin{tabular}{|c|c|c|c|c|}
        \hline
        \rowcolor[gray]{0.8}
        Word Address & Binary Address  & Tag  & Index & Hit/Miss\\
        \hline
        0x03 & 0000 0011 & 0 & 3 & M\\
        \hline
        0xb4 & 1011 0100 & b & 4 & M\\
        \hline
        0x2b & 0010 1011 & 2 & b & M\\
        \hline
        0x02 & 0000 0010 & 0 & 2 & M\\
        \hline
        0xbf & 1011 1111 & b & f & M\\
        \hline
        0x58 & 0101 1000 & 5 & 8 & M\\
        \hline
        0xbe & 1011 1110 & b & e & M\\
        \hline
        0x0e & 0000 1110 & 0 & e & M\\
        \hline
        0xb5 & 1011 0101 & b & 5 & M\\
        \hline
        0x2c & 0010 1100 & 2 & c & M\\
        \hline
        0xba & 1011 1010 & b & a & M\\
        \hline
        0xfd & 1111 1101 & f & d & M\\
        \hline
    \end{tabular}
    \label{tab:mytable}
\end{table}

b. For each of these references, identify the binary word address, the tag, the
index, and the offset given a direct-mapped cache with two-word blocks and a total size of
eight blocks. Also list if each reference is a hit or a miss, assuming the cache is initially
empty.

\begin{table}[h]
    \centering
    \begin{tabular}{|c|c|c|c|c|c|}
        \hline
        \rowcolor[gray]{0.8}
        Word Address & Binary Address  & Tag  & Index & offset & Hit/Miss\\
        \hline
        0x03 & 0000 0011 & 0 & 1 & 1 & M\\
        \hline
        0xb4 & 1011 0100 & b & 2 & 0 & M\\
        \hline
        0x2b & 0010 1011 & 2 & 5 & 1 & M\\
        \hline
        0x02 & 0000 0010 & 0 & 1 & 0 & H\\
        \hline
        0xbf & 1011 1111 & b & 7 & 1 & M\\
        \hline
        0x58 & 0101 1000 & 5 & 4 & 0 & M\\
        \hline
        0xbe & 1011 1110 & b & 6 & 0 & H\\
        \hline
        0x0e & 0000 1110 & 0 & 7 & 0 & M\\
        \hline
        0xb5 & 1011 0101 & b & 2 & 1 & H\\
        \hline
        0x2c & 0010 1100 & 2 & 6 & 0 & M\\
        \hline
        0xba & 1011 1010 & b & 5 & 0 & M\\
        \hline
        0xfd & 1111 1101 & f & 6 & 1 & M\\
        \hline
    \end{tabular}
    \label{tab:mytable}
\end{table}

\section*{Question 3}

Considering the address size of 64-bits, fill in the data for difference types of caches:

\begin{table}[h]
    \centering
    \begin{tabular}{|c|c|c|c|c|c|c|c|}
        \hline
        \rowcolor[gray]{0.8}
         & Blocks  & Data per Block  & Sets & Associativity-ways & Tag Bits & Index Bits & Offset Bits\\
        \hline
        Fully Associative Cache & 8 & 8 words & \-- & 8 & 59 & 0 & 5\\
        \hline
        Direct Mapped Cache & 16 & 8 words & \-- & 1 & 55 & 4 & 5\\
        \hline
        Associative Cache & 32 & 8 words & 4 & 8 & 57 & 2 & 5\\
        \hline
        Direct Mapped Cache & 64 & 8 words & \-- & 1 & 53 & 6 & 5\\
        \hline
        Associative Cache & 128 & 8 words & 32 & 4 & 54 & 5 & 5\\
        \hline
        Associative Cache & 256 & 8 words & 32 & 8 & 54 & 5 & 5\\
        \hline
        Fully Associative Cache & 512 & 8 words & \-- & 512 & 59 & 0 & 5\\
        \hline
        Direct Mapped Cache & 1024 & 8 words & \-- & 1 & 49 & 10 & 5\\
        \hline
        Associative Cache & 2048 & 8 words & 64 & 32 & 53 & 6 & 5\\
        \hline
        Direct Mapped Cache & 4096 & 8 words & 64 & 32 & 53 & 6 & 5\\
        \hline
    \end{tabular}
    \label{tab:mytable}
\end{table}

\section*{Question 5}

We are given 4 arrays of size 6. Each element in an array is of 32 bytes i.e., one word. Following
is the data stored in the array:
\\A = (25, 48, 43, 30, 47, 36)
\\B = (16, 29, 35, 38, 32, 41)
\\C = (24, 33, 5, 39, 10, 14)
\\D = (23, 7, 11, 44, 42, 22)
\\The array data is arranged in main memory as follows:

\begin{table}[h]
    \begin{tabular}{|c|c|}
        \hline
        00000 & $A[0]$ \\
        00001 & $A[1]$ \\
        00010 & $A[2]$ \\
        00011 & $A[3]$ \\
        00100 & $A[4]$ \\
        00101 & $A[5]$ \\
        00110 & \\
        00111 & \\
        01000 & $B[0]$ \\
        01001 & $B[1]$ \\
        01010 & $B[2]$ \\
        01011 & $B[3]$ \\
        01100 & $B[4]$ \\
        01101 & $B[5]$ \\
        01110 & \\
        01111 & \\
        10000 & $C[0]$ \\
        10001 & $C[1]$ \\
        10010 & $C[2]$ \\
        10011 & $C[3]$ \\
        10100 & $C[4]$ \\
        10101 & $C[5]$ \\
        10110 & \\
        10111 & \\
        11000 & $D[0]$ \\
        11001 & $D[1]$ \\
        11010 & $D[2]$ \\
        11011 & $D[3]$ \\
        11100 & $D[4]$ \\
        11101 & $D[5]$ \\
        11110 & \\
        11111 & \\
        \hline
    \end{tabular}
\end{table}

\vspace*{0.5 cm}
We are given a direct mapped cache which contains 8 blocks (each block will contain one word).
Insert the following elements in cache one by one and also mention whether it was a hit or a miss.
\\
\\Assume that the first block of the cache will be populated by the first element of the array and so
on. First insertion is already done so that you may get the idea.

\begin{tabular}{|m{15mm} | m{17mm} | m{8.5mm} |m{8.5mm} |m{8.5mm} |m{8.5mm} |m{8.5mm} |m{8.5mm} |m{8.5mm} |m{8.5mm} |}
    \hline
    \rowcolor[gray]{0.8}
    \raggedright\textbf{Data to \hspace*{3.5mm} be Inserted} &\raggedright \textbf{Hit/Miss} & \multicolumn{8}{|c|}{\textbf{Cache Index}} \\ \hline 
    \rowcolor[gray]{0.8}
    & & \hspace*{3mm}\textbf{0} & \hspace*{3mm}\textbf{1} & \hspace*{3mm}\textbf{2} & \hspace*{3mm}\textbf{3} & \hspace*{3mm}\textbf{4} & \hspace*{3mm}\textbf{5} & \hspace*{3mm}\textbf{6} & \hspace*{3mm}\textbf{7} \\ \hline
    A[0] & M & A[0] & & & & & & & \\ \hline
    A[1] & M & A[0] & A[1] & & & & & &\\ \hline
    A[2] & M & A[0] & A[1] & A[2] & & & & &\\ \hline
    A[1] & H & A[0] & \textcolor{red}{A[1]} & A[2] & & & & &\\ \hline
    A[5] & M & A[0] & A[1] & A[2]& & & A[5] & &\\ \hline
    B[5] & M & A[0] & A[1] & A[2] & & & \textcolor{blue}{B[5]} & &\\ \hline
    B[4] & M & A[0] & A[1] & A[2] & & B[4] & B[5] & &\\ \hline
    B[3] & M & A[0] & A[1] & A[2] & B[3] & B[4] & B[5] & &\\ \hline
    B[3] & H & A[0] & A[1] & A[2] & \textcolor{red}{B[3]} & B[4] & B[5] & &\\ \hline
    B[4] & H & A[0] & A[1] & A[2] & B[3] & \textcolor{red}{B[4]} & B[5] & &\\ \hline
    D[1] & M & A[0] & \textcolor{blue}{D[1]} & A[2] & B[3] & B[4] & B[5] & &\\ \hline
    D[2] & M & A[0] & D[1] & \textcolor{blue}{D[2]} & B[3] & B[4] & B[5] & &\\ \hline
    D[3] & M & A[0] & D[1] & D[2] & \textcolor{blue}{D[3]} & B[4] & B[5] & &\\ \hline
    D[4] & M & A[0] & D[1] & D[2] & D[3] & \textcolor{blue}{D[4]} & B[5] & &\\ \hline
    C[3] & M & A[0] & D[1] & D[2] & \textcolor{blue}{C[3]} & D[4] & B[5] & &\\ \hline
    C[2] & M & A[0] & D[1] & \textcolor{blue}{C[2]} & C[3] & D[4] & B[5] & &\\ \hline
    C[4] & M & A[0] & D[1] & C[2] & C[3] & \textcolor{blue}{C[4]} & B[5] & &\\ \hline
    C[2] & H & A[0] & D[1] & \textcolor{red}{C[2]} & C[3] & C[4] & B[5] & &\\ \hline
\end{tabular}

\vspace*{0.5 cm}
What is the Hit Ratio and the Miss Ratio in the above case?

\solution{
    Number of hits = 4
    \\Number of misses = 14
    \\Total accesses to the Memory = 18
    \\Hit Ratio = 4/18 = 0.222
    \\Miss Ratio = 14/18 = 0.778
}
    

\end{document}